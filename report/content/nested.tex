\section{Nested parallel forward and back projctions}
A first approach at forward and backprojection was to do a nested parallel version and using computed chunks of the system matrix.\\
A looped version of the forward projection with the system matrix cut in $steps$ chunks looks like this:

\begin{figure}[h]
\begin{lstlisting}[frame=single]
for step = 0; step < steps; step++
	A = getRays(raysperstep)
	for ray = 0; ray < raysperstep; row++
		acc = 0.0
		for p = 0; p<numpixels; p++
			acc+= A[ray][p]*image[p]
		FP[step*raysperstep+ray] = acc

this can be written in futhark like pseudo code as:

loop (output, step, raysperstep)
	let A = getRays raysperstep step
	let partresult = map (\row -> reduce (+) 0 <| map ( \i -> row[i]*vector[i] ) (iota (length row)) ) A
	in (output++partresult, step, raysperstep)
\end{lstlisting}
\caption{A looped version of the forward projection, where the raysperstep should be chosen such that the computations fit in the memory. $step*raysperstep$ should equal the total number of rows.}
\end{figure}
\begin{figure}[h]
\begin{lstlisting}[frame=single]
for step = 0; step < steps; step++
	A = getRays(raysperstep)
	AT = A.transpose()
	for p = 0; p<numpixels; p++
		acc = 0.0
		for ray = 0; ray<raysperstep; ray++
			acc+= AT[p][ray]*sinogram[ray]
		BP[p] += acc

this can be written in futhark like pseudo code as:

loop (output, step, raysperstep)
	let A = getRays raysperstep step
	let AT = transpose A
	let partresult = map (\row -> reduce (+) 0 <| map2 (*) row vect ) AT
	map2 (+) partresult output
	in (output, step, raysperstep)
\end{lstlisting}
  \caption{A looped version of the back projection, where the raysperstep should be the largest number possible such that the computations fit in the memory. $step*raysperstep$ should equal the total number of rows.}
\end{figure}
