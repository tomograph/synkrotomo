\section{Methodology}
In terms of derivation, we find it natural to base our work on the existing line drawing algorithm presented in the Futhark language book. This algorithm is a simple line rasterizer that will require significant modification in order to suit our needs. Once this algorithm has been modified, the search for optimization opportunities will begin. Each optimization will be implemented sequentially (and if possible in isolation), such that an estimate of their impact can be performed and charted.\\

Since the purpose of this project is to replace a relatively slow algorithm with a fast parallel algorithm, our main concerns are correctness and speed of the resulting algorithm. We will endeavour in a formal proof of correctness, but failing should we fail at this, we will rely on comparative verification by comparing the results of the parallel algorithm with the reference algorithm. Assuming both algorithms produce the same results in all test cases, an estimation of the likelihood of such an occurrence, in the presence of errors in the parallel algorithm, should be given. In either case, thorough unit testing should be performed, limiting the source of potential errors. 
\\

Once we are reasonably assured that our algorithm is correct, a comparative analysis of the parallel and reference algorithms should be performed. We should compare the speed of the two algorithms on a typical data set, but also of varying sizes and dimensions of data sets, such that a clear picture of the  strengths and weaknesses of each algorithm are presented. Finally, assuming that optimization opportunities have been found and realized, an estimation and presentation of the effect of these should be presented. 



